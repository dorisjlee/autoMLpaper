\section{Study Design}
Our paper seeks to understand how auto-ml users incorporate auto-ml tools in their existing machine learning workflow and their perceptions of the tools. 

\subsection{Recruitment}
We chose to focus on users who have used auto-ml for real-world use cases, in a diverse set of application domains. The participants either indicated that they had experience using at least one of the tools that we listed in the recruitment survey or they have used other tools that our team verified to be auto-ml [TODO: what is our criteria in defining auto-ml? - that the tool at least does automated model selection and hyperparameter tuning?]. 

We found participants in three ways: First, our personal connections (x participants). 
Second, we searched mentions of specific automated machine learning tools on twitter and invited the twitter users to fill out the recruitment form which asks whether they have experience using automated machine learning and sign up for an interview slot if the answer is affirmative (x participants). 
Third, we posted a recruitment message various relevant mailing lists and slack channels both in our university and in the industry. 

\subsection{Participants}
 We interviewed a total of 20 people (x male, x female; 3 White/Caucasian, 3 Asian, 2 South Asian, 1 Hispanic; aged 18 to 30, M = 21). 

Our interview participants had an average of x years experience in machine learning and x years of experience in programming. 

[TODO: add more quantitative measures this weekend, i.e. job titles, business sectors]

\subsection{Interview Procedure}

We began our study by conducting semi-structured interviews with users of automated machine learning for real-life use cases. This enabled us to establish a basic understanding of their machine learning workflow, the reasons why they choose to use auto-ml and their perceptions of auto-ml. We interviewed with 20 automated machine learning users. 

We conducted the interviews from October 2019 to March 2020.The Interviews consisted three stages. First, the participants were asked to describe their job functions and their experience levels with programming and machine learning.
Second, participants were asked to describe their experience in developing machine learning models and the challenges they face. Third, interviewers were asked to describe their experience in using automated machine learning tools and their perceptions of the tool(s). 

Interviews were either in person (N =x) or conducted remotely (N=x). Each interview lasted for about one hour. Every participant is reimbursed with $\$15$ for their time and insights.

We audio recorded and transcribed all interviews. Our interviews were semi-structured and centered around the following questions: [TODO: add more details depending on the findings]
\begin{itemize}
\item What is your current machine learning workflow with and without automated machine learning?
\item How do they perceive of automated machine learning?
\item If they could, what would automated machine learning users change about machine learning packages and automated machine learning tools?
\end{itemize}


% Understanding users' viewpoints, values, priorities, work practices and strategies is essential to human-in-the-loop machine learning in ways that matter to the people with the most at stake. Here we focus on real-life auto-ml users and ask: Who are the users of automated machine learning? How do these users integrate automated machine learning into their existing machine learning work practices? How to optimize collaboration between humans and automated machine learning?

% At the end of our study, we sought to complement our findings with a survey survey of practitioners who opted to not use auto-ml to understand the obstacles for adoption. 

% We also gathered data from artifacts of automated machine learning workflows such as notebooks, screenshot of notebooks 2) Collaborative and iterative data analysis: Our team met regularly in the research process and analyzed and discussed the data together \cite{beebe1995basic}. 

% \subsection{Data Gathering}
% We gathered data in two main ways: directly in interviews and surveys and indirectly through content analysis. This enabled us to verify our findings with more confidence by both eliciting perceptions from people and sometimes probing them to go deeper, as well as analyzing their work products in context. 


% \subsubsection{Content Analysis}
% Our second source of data was auto-ml artifacts.

