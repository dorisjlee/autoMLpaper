%
% The first command in your LaTeX source must be the \documentclass command.
\documentclass[acmsmall]{acmart}

%
% defining the \BibTeX command - from Oren Patashnik's original BibTeX documentation.
\def\BibTeX{{\rm B\kern-.05em{\sc i\kern-.025em b}\kern-.08emT\kern-.1667em\lower.7ex\hbox{E}\kern-.125emX}}
    
% Rights management information. 
% This information is sent to you when you complete the rights form.
% These commands have SAMPLE values in them; it is your responsibility as an author to replace
% the commands and values with those provided to you when you complete the rights form.
%
% These commands are for a PROCEEDINGS abstract or paper.
\copyrightyear{2018}
\acmYear{2018}
\setcopyright{acmlicensed}
\acmConference[Woodstock '18]{Woodstock '18: ACM Symposium on Neural Gaze Detection}{June 03--05, 2018}{Woodstock, NY}
\acmBooktitle{Woodstock '18: ACM Symposium on Neural Gaze Detection, June 03--05, 2018, Woodstock, NY}
\acmPrice{15.00}
\acmDOI{10.1145/1122445.1122456}
\acmISBN{978-1-4503-9999-9/18/06}
%\newcommand{\edit}[1]{{\color{blue!100}#1}}
%
% These commands are for a JOURNAL article.
%\setcopyright{acmcopyright}
%\acmJournal{TOG}
%\acmYear{2018}\acmVolume{37}\acmNumber{4}\acmArticle{111}\acmMonth{8}
%\acmDOI{10.1145/1122445.1122456}

%
% Submission ID. 
% Use this when submitting an article to a sponsored event. You'll receive a unique submission ID from the organizers
% of the event, and this ID should be used as the parameter to this command.
%\acmSubmissionID{123-A56-BU3}

%
% The majority of ACM publications use numbered citations and references. If you are preparing content for an event
% sponsored by ACM SIGGRAPH, you must use the "author year" style of citations and references. Uncommenting
% the next command will enable that style.
%\citestyle{acmauthoryear}

%
% end of the preamble, start of the body of the document source.
\begin{document}

\setcopyright{acmlicensed}
\acmJournal{PACMHCI}
\acmYear{2019} \acmVolume{3} \acmNumber{CSCW} \acmArticle{219} \acmMonth{11} \acmPrice{15.00}\acmDOI{10.1145/3359321}
%
% The "title" command has an optional parameter, allowing the author to define a "short title" to be used in page headers.
\title{AutoML User Survey}

%
% The "author" command and its associated commands are used to define the authors and their affiliations.
% Of note is the shared affiliation of the first two authors, and the "authornote" and "authornotemark" commands
% used to denote shared contribution to the research.
% \author{Anonymized for review}

%%
%% The "author" command and its associated commands are used to define
%% the authors and their affiliations.
%% Of note is the shared affiliation of the first two authors, and the
%% "authornote" and "authornotemark" commands
%% used to denote shared contribution to the research.
\author{}
% \authornote{Both authors contributed equally to this research.}
\email{eva.wu@berkeley.edu}
% \orcid{1234-5678-9012}
% \author{G.K.M. Tobin}
% \authornotemark[1]
% \email{webmaster@marysville-ohio.com}
\affiliation{%
  \institution{University of California, Berkeley}
%   \streetaddress{P.O. Box 1212}
  \city{Berkeley}
  \state{California}
  \country{USA}
%   \postcode{94720}
}

\author{}
\affiliation{%
  \institution{University of California, Berkeley}
%   \streetaddress{1 Th{\o}rv{\"a}ld Circle}
  \city{Berkeley}
  \country{USA}}
\email{epedersen@berkeley.edu}

\author{}
\affiliation{%
  \institution{University of California, Berkeley}
%   \streetaddress{1 Th{\o}rv{\"a}ld Circle}
  \city{Berkeley}
  \country{USA}}
\email{nsalehi@berkeley.edu}

%
% By default, the full list of authors will be used in the page headers. Often, this list is too long, and will overlap
% other information printed in the page headers. This command allows the author to define a more concise list
% of authors' names for this purpose.
% \renewcommand{\shortauthors}{Wu, et al.}
% \\renewcommand{\\shortauthors}{Eva Yiwei Wu, Emily Pedersen, \\& Niloufar Salehi}
\renewcommand{\shortauthors}{Eva Yiwei Wu}
%
% The abstract is a short summary of the work to be presented in the article.
\begin{abstract}

\end{abstract}

%
% The code below is generated by the tool at http://dl.acm.org/ccs.cfm.
% Please copy and paste the code instead of the example below.
%

 \begin{CCSXML}
<ccs2012>
<concept>
<concept_id>10003120.10003130.10003131.10003570</concept_id>
<concept_desc>Human-centered computing~Computer supported cooperative work</concept_desc>
<concept_significance>500</concept_significance>
</concept>
</ccs2012>
\end{CCSXML}

\ccsdesc[500]{Human-centered computing~Computer supported cooperative work}

%
% Keywords. The author(s) should pick words that accurately describe the work being
% presented. Separate the keywords with commas.
\keywords{Algorithmic Persona; YouTube; Metaphor; Folk Theories; Content Creators}

%
% A "teaser" image appears between the author and affiliation information and the body 
% of the document, and typically spans the page. 
%%\begin{teaserfigure}
%%  \includegraphics[width=\textwidth]{sampleteaser}
%%  \caption{Seattle Mariners at Spring Training, 2010.}
%%  \Description{Enjoying the baseball game from the third-base seats. Ichiro Suzuki preparing to bat.}
%%  \label{fig:teaser}
%%\end{teaserfigure}

%
% This command processes the author and affiliation and title information and builds
% the first part of the formatted document.
\maketitle

\section{Introduction}
Automated Machine Learning (auto-ml) techniques have been recently introduced to simplify and accelerate the process of developing machine learning models by automating tasks such as data cleaning, model selection, and hyperparameter tuning. The promise of auto-ml tools is to make machine learning more accessible for citizen data analysts by providing off-the-shelf solution for users with less technical background about machine learning. Auto-ml tools also promise machine learning experts the relief from the mechanical tasks such as hyperparameters tuning. In this study, we aim to examine where Auto-ml fits within a typical machine learning workflow, the perception of auto-ml from users of auto-ml, and auto-ml practitioners’ usage behavior. 

% We ask 
% \begin{itemize}
%     \item RQ1: Who are the auto-ml users and non-auto-ml users? 
%     \item RQ2: How do auto-ml users perceive of auto-ml?
%     \item RQ3: What factors lead one to choose manual ML over auto-ml?
%     \item RQ4: What improvements in auto-ml would auto-ml users like to see? 
%     \item RQ5: What is the impact of auto-ml on AI explanability? Does auto-ml make ML abstruse for the interpretation of clients of data science teams?
%     \item RQ6: What is the ideal integration between human and Automated Machine Learning? 
% \end{itemize}

We contribute to this research space by engaging with users of Auto-ml to inform the design of systems for individuals who can most benefit from Auto-ml. In addition, we fill in the gap in the understanding of how auto-ml can be designed for data scientists to trust it enough to use it in practice by providing empirical evidence of perception of explainability, transparency and trust in auto-ml.

This is the first in-depth study of auto-ml practitioner to uncover their latent needs and wants. We reject the assumption that auto-ml design is to automate human workers, but rather to enhance humans' work. Here we extend prior research in calling for collaboration of human and AI and for AI to take on more an "augmentor" role rather than an "automator" role. The insights from the study will help guide the design of human-centered auto-ml systems. 


% In particular, we include in our study samples citizen data scientists, data scientist novices who are auto-ml users because their perspectives are lacking in the academic literature, despite being a major target group of auto-ml that promises to democratizes data science. 
\section{Related Work}
Our work builds on top of three major areas of prior work: \automl systems, human-centric view of \automl and user studies of data science practitioners. 

\subsection{Automated Machine Learning (\automl) Systems}
\par % state of automl
\automl technology is fast-growing and still in very much in its nascency, with a diverse set of commercial and academic offerings. Interfaces of \automl tools ranges from GUI-based to programmatic, catering to different groups of users, including machine learning engineers, data scientists, and non-programming business users. \automl tools vary in the degree of input customizability they offer, as well as their output model transparency. For example, some systems allow users to specify a range of models to select from (CITE) or metrics to optimize (CITE), while others (CITE) only allow users to specify the problem objective, such as classification or regression. In terms of the model output, some tools (CITE) exposes the model in its entirety to the user so that they can be fine-tuned further, while others (CITE google automl) simply return a monolithic black-box model that serves predictions. While \automl tools have traditionally focused on automating the modelling phase of in the machine learning lifecycle, there has been a growing suite of \automl tools that aims to democratize the end-to-end machine learning process, from data pre-processing to model building to post-processing. 
%These tools offers varying levels of support for all stages of the machine learning workflow, while others offer partial solution that only address one or two phases of the machine learning workflow. They also vary in their underlying optimization techniques, transparency and customizability. 
However, despite the excitement around this emerging technology, there have been little evaluation on how such tools are used in practice. In this paper, we sought to understand the adoption of \automl tools, their usage, and the current bottlenecks that users are facing with these tools.

% ---- community understanding 

% The intention of the tools is to relieve practitioners of the repetitive and time-consuming tasks in the machine learning process while providing solutions at speed and spare human resource time for other critical tasks that are not easily automated.


% Data pre-processing includes ETL, data cleaning, and exploratory data analysis. Model building involves model selection, hyperparameter tuning, and model ensembling. Post-processing include performance evaluation and insight and report generation. 

% (however, many practitioners enjoy the practices. refer to 203.)

% These tools largely fall into three categories based on their accessibility: open source packages such as \cite{Feurer:autosklearn,TPOT:Olson, auto-keras}, commercial services such as Amazon Sagemaker, Google AutoML, H2O Driveless AI, DataRobot and enterprise internal tools such as FBLearner. These tools can be categorized based on their comprehensiveness: some offer end-to-end solution with various levels of support for all stages of the machine learning workflow, while others offer partial solution that only address one or two phases of the machine learning workflow. They also vary in their underlying optimization techniques, transparency and customizability. The lack of control and opacity in many of these \automl tools have spurred interest in research around human-oriented \automl.

\subsection{Human-Centric \automl}
Some prior work has investigated the human-centric perspective of automated machine learning, advocating for a collaborative approach between human and automated machine learning \cite{Lee2019AHP,wang2018atmseer, gilHGML}, in which \automl augments human practitioners in speed and accuracy and human practitioners guide \automl using their domain knowledge. Taking a human-centric approach, these prior work explored data scientists' perceptions of  \automl, including concepts such as transparency, trust and interpretability and proposed design recommendations to increase human trust in \automl and the usability of tools through visual analytics \cite{wang2018atmseer, cashman et al snowcast}.   

Gil et al. proposed human-guided machine learning (HGML) as a hybrid approach to incorporate human knowledge to augment and improve the performance of \automl and argued that complete automation is not ideal for all use cases (Gil et al., 2019). In their study, Gil et al. analyzed machine learning workflows described in two academic publications from two different disciplines to complement their understanding of machine learning user behaviors and to inform their design recommendations for HGML systems.

% Snowcat \cite{Cashman et al., 2018} provides visualization of machine learning models used in \automl with the aim to shed light on the black boxes of \automl. Cashman et al. built the Snowcat system based on theoretical framework and architecture without using any qualitative methods. 

Wang et al. designed ATMSeer \cite{wang2018atmseer}, a visual analytics tool that enables search space refinement, computational budget adjustment, and model selection reasoning.  They designed ATMSeer based on feedback from semi-structured interviews with six machine learning practitioners in which they sought to understand the opportunities exist to improve the process of how machine learning practitioners choose machine learning models.  

Wang et al. echo previous research, arguing that complete automation might not be desirable in all use cases and highlighting the multifarious objectives of data scientists' work \cite{DakuoCSCW2019}. Their study also revealed the different relationships between human and \automl, such as \automl as a collaborator, teacher and a data scientist. They concluded their paper by outlining design recommendations that allow \automl to augment human data scientists through collaboration, for example, integrating xAI techniques in \automl user interfaces to provide answers in some of the "why" questions to increase trust and giving data scientists the full control over the final choices of the entire pipeline and design these features for model interpretation for users from diverse backgrounds. 

Drozdal et al. studied trust in \automl and found that increasing transparency increases trust, however it depends on the users' purposes and context. \cite{DrozdalTrustAutoml}

\subsection{Empirical Studies of Data Practitioners}
Recent research has also studied data science and machine learning practitioners' workflow without automated machine learning. \cite{}. Amershi et al. surveyed software engineers about their work practices building and integrating machine learning into software and services. They conducted interviews to gather insights that informed their research questions and developed a wide-scale survey about the identified topics. They identified a challenge in machine learning is that iterating on models is time and labor extensive. Another set of related work studied non-experts of machine learning. For example, Yang et al. used interviews and survey and investigated how non-experts build machine learning solutions in real life. \cite{YangNonExpert} Kandel et al. studied analysts in the industry to understand their work process, struggles and potential solutions. \cite{Kandel}

To our knowledge, no research has drawn all their study subjects from users of automated machine learning who have experience using \automl in real-world applications and with an established relationship with \automl tools. Many existing studies focus only on the modelling phase in the machine learning process. Our study contributes to existing work by studying users who have used \automl for real-world applications and we sought to understand their work practices across all stages of the machine learning process and the humans involvement in their current workflow. 
Our study answers the following questions:[TODO: add more details depending on the findings]
\textbf{RQ1}: Who are the users of Automated Machine Learning tools?
\textbf{RQ2}: What are their current work practices and what strategies do they employ to integrate automated machine learning into their existing workflow?
\textbf{RQ3}: What are their perceptions of \automl and what are the affect of \automl on \automl users?


% \subsection{Our Contribution to Prior Work}
% Prior work has leveraged controlled experiments or have studied machine learning practitioners. 


% We reveal the multifarious motivations of the practicing machine learning: enjoyment (similar to readings in 203 about professional values of nurses who want to provide care.) or not to simply gain the best accuracy score [is there anything besides what's covered in Dakuo's uncover relationship in the data?] [This begs the question, what do data scientists use machine learning for?]










\section{Study Design}
Our paper seeks to understand how auto-ml users incorporate auto-ml tools in their existing machine learning workflow and their perceptions of the tools. 

\subsection{Recruitment}
We chose to focus on users who have used auto-ml for real-world use cases, in a diverse set of application domains. The participants either indicated that they had experience using at least one of the tools that we listed in the recruitment survey or they have used other tools that our team verified to be auto-ml [TODO: what is our criteria in defining auto-ml? - that the tool at least does automated model selection and hyperparameter tuning?]. 

We found participants in three ways: First, our personal connections (x participants). 
Second, we searched mentions of specific automated machine learning tools on twitter and invited the twitter users to fill out the recruitment form which asks whether they have experience using automated machine learning and sign up for an interview slot if the answer is affirmative (x participants). 
Third, we posted a recruitment message various relevant mailing lists and slack channels both in our university and in the industry. 

\subsection{Participants}
 We interviewed a total of 20 people (x male, x female; 3 White/Caucasian, 3 Asian, 2 South Asian, 1 Hispanic; aged 18 to 30, M = 21). 

Our interview participants had an average of x years experience in machine learning and x years of experience in programming. 

[TODO: add more quantitative measures this weekend, i.e. job titles, business sectors]

\subsection{Interview Procedure}

We began our study by conducting semi-structured interviews with users of automated machine learning for real-life use cases. This enabled us to establish a basic understanding of their machine learning workflow, the reasons why they choose to use auto-ml and their perceptions of auto-ml. We interviewed with 20 automated machine learning users. 

We conducted the interviews from October 2019 to March 2020.The Interviews consisted three stages. First, the participants were asked to describe their job functions and their experience levels with programming and machine learning.
Second, participants were asked to describe their experience in developing machine learning models and the challenges they face. Third, interviewers were asked to describe their experience in using automated machine learning tools and their perceptions of the tool(s). 

Interviews were either in person (N =x) or conducted remotely (N=x). Each interview lasted for about one hour. Every participant is reimbursed with $\$15$ for their time and insights.

We audio recorded and transcribed all interviews. Our interviews were semi-structured and centered around the following questions: [TODO: add more details depending on the findings]
\begin{itemize}
\item What is your current machine learning workflow with and without automated machine learning?
\item How do they perceive of automated machine learning?
\item If they could, what would automated machine learning users change about machine learning packages and automated machine learning tools?
\end{itemize}


% Understanding users' viewpoints, values, priorities, work practices and strategies is essential to human-in-the-loop machine learning in ways that matter to the people with the most at stake. Here we focus on real-life auto-ml users and ask: Who are the users of automated machine learning? How do these users integrate automated machine learning into their existing machine learning work practices? How to optimize collaboration between humans and automated machine learning?

% At the end of our study, we sought to complement our findings with a survey survey of practitioners who opted to not use auto-ml to understand the obstacles for adoption. 

% We also gathered data from artifacts of automated machine learning workflows such as notebooks, screenshot of notebooks 2) Collaborative and iterative data analysis: Our team met regularly in the research process and analyzed and discussed the data together \cite{beebe1995basic}. 

% \subsection{Data Gathering}
% We gathered data in two main ways: directly in interviews and surveys and indirectly through content analysis. This enabled us to verify our findings with more confidence by both eliciting perceptions from people and sometimes probing them to go deeper, as well as analyzing their work products in context. 


% \subsubsection{Content Analysis}
% Our second source of data was auto-ml artifacts.


\section{Study Analysis}
We analyze our study data agnostic of specific auto-ml tools but we aggregated our findings accounting for the categorical differences among the tools in our analysis of usage behaviors and perceptions, for example, open source vs. commercial tools.

We engaged in an iterative and collaborative process of inductive coding to extract common themes that repeatedly came up in our data. After completing the interviews, we met weekly and discussed themes and concepts as we continued our fieldwork. We used Dedoose, an online tool for open coding, to map data onto these categories. Each of first three authors independently coded the entire data. We conducted a categorization exercise in which we physically laid out themes and relevant quotes into emerging categories. Some of our initial categories included [TODO]. Through the open coding phase, the category of [TODO...] was the most pervasive, occurring in all of our transcripts.
\section{Discussion: Design Implications}
\section{Conclusion}

\bibliography{refs}
\bibliographystyle{ACM-Reference-Format}

\received{April 2019} 
\received[revised]{June 2019}
\received[accepted]{August 2019}

\end{document}
